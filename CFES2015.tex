\documentclass[12pt,oneside,english]{scrbook}
\usepackage[]{graphicx}
\usepackage[]{color}
\usepackage{framed}
\usepackage{alltt}
\usepackage[T1]{fontenc}
\usepackage[latin9]{inputenc}
\setcounter{secnumdepth}{3}
\setcounter{tocdepth}{3}
\usepackage{babel}
\usepackage{array}
\usepackage{longtable}
\usepackage{booktabs}
\usepackage{url}
%\usepackage[unicode=true,pdfusetitle,
% bookmarks=true,bookmarksnumbered=false,bookmarksopen=false,
% breaklinks=true,pdfborder={0 0 1},backref=false,colorlinks=true]
% {hyperref}
%\hypersetup{urlcolor=blue,linkcolor=blue}
\usepackage{colortbl}
\usepackage{ragged2e}
\usepackage{verbatim}
\usepackage[pyfuture=none]{pythontex}  

% The following preamble allows me to use biblatex to generate
% multiple reference sections using keyword filters
\usepackage[backend=biber,style=authoryear,dashed=false,maxnames=99]{biblatex}
\addbibresource{aubrey.bib}
\nocite{*}

% The following code enumerates each bibliography.
\defbibenvironment{bibliography}
{\enumerate{}
{\setlength{\leftmargin}{\bibhang}%
\setlength{\itemindent}{-\leftmargin}%
\setlength{\itemsep}{\bibitemsep}%
\setlength{\parsep}{\bibparsep}}}
{\endenumerate}
\item{}

\usepackage{hyperref}
\hypersetup{
  colorlinks   = true, %Colours links instead of ugly boxes
  urlcolor     = blue, %Colour for external hyperlinks
  linkcolor    = blue, %Colour of internal links
  citecolor    = red   %Colour of citations
}

% The following two lines prevent long urls from extending
% beyond the right margin
\setcounter{biburllcpenalty}{7000}
\setcounter{biburlucpenalty}{8000}


\begin{document}

%\begin{pycode}
%import crbtechreportlongtable
%create_longtable()
%\end{pycode}

\title{CFES 2015}
\author{Aubrey Moore, Ph.D.\\
Associate Professor / Extension Entomologist}
\maketitle

I was hired by the University of Guam on October 1, 2003 under a limited-term,
split appointment (50\% extension and 50\% research). On June 26,
2008, I started a tenure-track appointment as extension entomologist
with the academic rank of Assistant Professor. I am a faculty member
of the Environmental Science Graduate Program and a member of the
Western Pacific Tropical Research Center. At the end of the 2012 fall
term I applied for tenure and promotion and received both.

This report documents my activities from June 2014 through the present.
My current faculty role allocation is as follows:
\begin{itemize}
	\item 51\% Extension and Community Activities 
	\item 34\% Creative/Scholarly Activity or Research 
	\item 15\% University and Community Service
\end{itemize}

\textbf{Note to Reader:}

This report is available as an electronic document in PDF format on
my website at \url{http://guaminsects.net/anr/content/cfes-2015-report-aubrey-moore}.
Because this is not a public document, you will need to identify yourself
to gain access by entering user name: \textbf{UogAdministrator} and
password: \textbf{let\_me\_in}. 

If you are reading the PDF version of the report, you will be able
to follow hypertext links to documents I have referenced. 

\tableofcontents{}

\listoftables

\chapter{Extension and Community Activities}

\section{Diagnostic Services}

As an extension entomologist, a major part of my job is providing
insect identification and pest control recommendations to a diverse
clientele including commercial growers, gardeners, householders, GovGuam
and federal agency personnel, and University of Guam colleagues. Most client contacts are initiated by
a phone call or a visit by the client to the ANR office. In many cases
identification and pest control recommendations require a site visit
by me and/or extension associates to collect samples and define the
problem. The number of extension calls requiring my assistance averages
approximately three per day. 

\subsection{Detection and Documentation of Invasive Species}

\subsection{Insect Identification Service for USDA-APHIS / Guam Customs and Quarantine
Agency}

I am often called upon to identify insect specimens intercepted the
Guam Customs and Quarantine Agency. USDA-APHIS has certified me for
this service and has provided a very official looking badge to impress
people with. (However, it is not quite as impressive as Dr. Millers
bright red badge for getting onto the airport runways.)

\subsection{Impediments}
\begin{itemize}
\item Taxonomic skills.
\item Lab facilities.
\item Microscope.
\end{itemize}

\section{University of Guam Insect Collection}

The UOG insect collection is a valuable reference collection for extension
entomology, teaching and research. I am a member of the board of directors
for the collection and I work with Dr. Ross Miller to curate and catalog
this collection. 

To increase my knowledge of collection management, I attend the annual
meetings of the Entomological Collections Network, which are typically
held in conjunction with annual meetings for the Entomological Society
of America.

I have a professional goal of building an online website to share
all available information on Micronesian insects. This will include
specimen level information for the collection complete with digital
images and literature references. I built a digital catalog for the
collection is using the BioLink Biodiversity Information Management
System from CSIRO, Australia. The catalog currently contains 29,200
specimen records. BioLink is currently being redeveloped as an open
source project (http://code.google.com/p/biolink/). In am an active
collaborator in this project. In July 2012 I published an article
entitled \emph{Hosting a Biolink Database in the Amazon Web Services
Cloud (EC2)} on the project's wiki (http://code.google.com/p/biolink/wiki/BioLinkEC2).

I have built and evaluated two websites for serving information on
Micronesian insect biodiversity, including specimen level data from
the collection. One is a Drupal content management system template
called LifeDesk provided the Encyclopedia of Life Project and the
other is a similar template called ScratchPads provided by the Museum
of Natural History in London. I am honored to have been selected as
an advocate for ScratchPads as part of the project's Ambassadors program
(http://scratchpads.eu/locate-scratchpad-ambassadors). Further information
on my websites is provided in the Creative/Scholarly Activities section
(\ref{sec:Web-Sites-Designed}). 

In March 2014 I travelled to Honolulu to attend the Biodiversity Collections
Digitization in the Pacific workshop sponsored by the Integrated Digitized
Biocollections (IDigBio). I made an oral presentation entitled \href{Evaluation of a Scratchpad Template as an https://www.idigbio.org/wiki/images/a/aa/Scratchpads_iDigBio-part1.pdf}{Evaluation of a Scratchpad Template as an Online Database for the University of Guam Insect Colletion}at
this workshop. 

In May 2014 I met with Dr. Bob Foottit at the Canadian National Insect
Collection in Ottawa to discuss progress and future directions for
the UOG collection. Dr. Foottit is a member of the board of directors
for the UOG Insect collection. 

\section{Guam Coconut Rhinoceros Beetle Eradication Project}

This is currently my largest and most time consuming project. 

The coconut rhinoceros beetle (CRB) was first detected on Guam in
the Tumon Beach hotel area on September 11, 2007. CRB is avery serious
pest of coconut palms. Adult beetles may kill coconuts and other palms
when they bore into the crowns to feed on sap. When CRB invaded Palau
during the Second World War, it killed about half of all coconuts
through the islands and totally exterpated the coconut palm from some
of them. A delimitation survey indicated that the Guam infestation
was limited to Tumon Bay and the adjacent Faifai Beach. In consultation
with the Guam Department of Agriculture (GDOA), USDA-APHIS, and USDA-Forest
Survey, it was decided to launch an eradication project. 

I wrote the original eradication plan (available on-line at \url{http://guaminsects.net/uogces/kbwiki/index.php?title=Coconut_Rhinoceros_Beetle_Eradication_Plan})
and this was funded by USDA and local funds. USDA provided funds under
the condition that the poject was to be run under an Incident Command
System with the USDA-APHIS Guam Port Director as the federal commander,
and the GDOA Director, or designee, as the local commander.

My original role was to provide scientific/technical support for the
project, with the Guam Department of Agriculture (GDOA) providing
project management with assistance from USDA-APHIS and USDA-Forest
Service. However, it soon became apparent that GDOA had serious bureaucratic
impediments which prevented hiring staff and procuring supplies and
equipment within a reasonable time frame. The eradication project
directors, with the consent of the Dean, agreed to run project staffing,
procurement, and fiscal management through the University. As a result,
my role was expanded to include much of the project management. I
am currently managing two grants which fund the project and supervise
about 15 temporary employees. Report writing on current grants and
proposal writing to keep the project in business occupies much of
my time. 

In December 2013, an infestation of CRB was detected on Hickam Air
Force Base on Oahu. Roland Quitugua and myself were recruited as subject
matter experts and spent a week in Honolulu advising an incident command
team set up by APHIS. Later, we were both added to a national technical
working group for CRB. My acitivities in support of the Hawaii CRB
Eradication project are detailed in the Regional Collaboration section
\ref{sub:Hawaii CRB}. 

\subsection{Activities:}
\begin{enumerate}
\item \textbf{Biweekly Planning Meetings.} This project is run as an incident
command system. I attend biweekly planning meetings as a program manager.
\item \textbf{Conference Calls.} These teleconferences are with stakeholders
and advisers in USDA APHIS and USDA Forest Service. These agencies
are funding the project. Until recently calls were biweekly. They
are now monthly.
\item \textbf{Grant Writing. }During the past 2 years, the Guam CRB Eradication
Project has been almost entirely funded from 12 grants for which I
wrote proposals and act as principal investigator. These grants are
listed in the Creative / Scholarly Activity section.
\item \textbf{Report Writing.} All grants supporting the Guam CRB Eradication
Project require regular reporting. 
\item \textbf{Project Websites. }I have endeavored to share and archive
data and information associated with the Guam CRB Eradication Project
on-line. Prior to May 2009, I used a wiki site at \url{http://www.guaminsects.net/uogces/kbwiki/index.php?title=Oryctes_rhinoceros}.
Afterwards, I used a Drupal site at \url{http://www.guaminsects.net/anr/category/miscellaneous/coconut-rhinoceros-beetle}.
\item \textbf{Project Database. }Trapping data from a network of about 1200
traps, detections of CRB grubs or adults, and observations of CRB
defoliation and bore holes are entered daily into a web-based georeferenced
MySQL database which I designed. Data from this database is publicly
accessible from a web page at \url{http://www.guaminsects.net/anr/content/public-access-data-collected-guam-coconut-rhinoceros-beetle-eradication-project}.
Links on this page enable the user to view trap catch data as a spatiotemporal
display using a Google Earth animation or a chart of monthly totals.
I use this system to produce monthly surveillance reports. 
\item \textbf{Scientific/technical Support.} I do applied research in support
of the Guam CRB Eradication Project. Results of this research is provided
in a series of technical reports. 
\item \textbf{Collaboration.} I have formed two collaborative research groups
to do applied research aimed at controling CRB damage. Dr. Sean Marshall
and Dr. Trevor Jackson at AgResearch New Zealand collaborate with
me on biological control using oryctes nudivirus (OrNV) and CRB population
genetics. Dr. Matthew Siderhurst and Dr. Eric Jang of USDA-ARS-PBARC
collaborate with me on CRB trap improvement. 
\end{enumerate}

\subsection{Impediments}
\begin{itemize}
\item My heavy workload does not permit enough time to prepare research
results for publication in scientific journals.
\end{itemize}

\section{Western Plant Diagnostics Network}

I am the UOG coordinator for WPDN. This organization provides financial
support for ANR's Plant Diagnostic Laboratory, offers First Detector
Training workshops, and organizes identification workshops for important
pest groups. As coordinator, I am required to organize First Detector
Training workshops, attend monthly conference calls, attend annual
meetings, and provide reports.

\section{Guam Invasive Species Advisory Committee (GISAC)}

I am an active, founding member of this informal group of Guam's biologists
which meets irregularly about 6 times per year to discuss invasive
species and what can be done to keep them out and mitigate the effects
of those that do invade the island. I worked with Dr. Russell Campbell
and Diane Vice to develop an emergency response plan for invasive
species detected on Guam.

A wiki site which I built for for GISAC was quickly adopted by the
Western Micronesia Regional Invasive Species Council at \url{http://guaminsects.net/gisac/index.php?title=Main_Page}.
(Evidence 1.6)

\section{Public Outreach (Guest lectures, presentations, interviews)}

During the reporting period I was interviewed numerous times by newspaper
reporters, radio talk show hosts, and television news reporters (Table
\ref{outreachTable}). Most, but not all involved questions about
the Guam coconut rhinoceros beetle eradication project. I produced
several fact sheets and articles for public print media during my
two years as extension entomologist year and also published a lot
of content on various websites. I have evaluated several current technologies
for building a web presence for the Agriculture and Natural Resources
Unit and the Drupal content management system seems to be a good fit.
This allows us to publish information for public access while keeping
some documents private for internal use only. My print and online
output are discussed in more detail in the Creative/Scholarly Activity
section. 

\section{Public Outreach (Internet)}

I maintain a website for the the UOG Cooperative Extension Service's
Agriculture and Natural Resources Program at \url{http://guaminsects.net/ANR}.
I frequently post blog articles of public interest to this site (Table
\ref{anrBlogTable}). I also maintain a website at \url{http://guaminsects.myspecies.info}
which is intended to facilitate sharing information on insects in
Micronesia. I frequently submit blog articles to this website which
are of interest to entomologists (Table \ref{myspeciesBlogTable}). 

\section{Regional Collaboration }

\subsection{Regional Invasive Species Council Website}

I maintain a website for the Western Micronesia Regional Invasive
Species Council (RISC) at \url{http://www.guaminsects.net/gisac/}.
I attend RISC meetings whenever they are held on Guam and I make presentations
at these meetings.

\subsection{Insect Diagnostics for Micronesia}

I am often contacted with requests for help with identifying pests
from throughout Micronesia and suggesting solutions to the problems
they cause. I expect this workload to increase because the number
of practicing PhD level entomologists in Micronesia has dropped from
9 to 3 within the last decade.

\subsection{\label{sub:Hawaii CRB}Support for the Hawaii Coconut Rhinoceros
Beetle Eradication Project}

In December 2013, an infestation of CRB was detected on Hickam Air
Force Base on Oahu. Roland Quitugua and myself were recruited as subject
matter experts and spent a week in Honolulu advising an incident command
system (ICS) team set up by APHIS. Later, we were both added to a
national technical working group (TWG) for CRB. I built and maintain
an online, full-text bibliographic for use by the TWG at \url{http://guaminsects.myspecies.info/CRB_biblio}.

Frequent requests for scientific/technical information from the ICS,
TWG and Hawaii Department of Agriculture (several queries per week)
has significantly increased my workload over the past several months.

\pagebreak{}

\chapter{Creative/Scholarly Activities or Research}

\section{Refereed Scientific Journal Articles}

\begingroup
\let\clearpage\relax
\printbibliography[heading=none,	keyword=cfes2015, keyword=journal]
\endgroup

\section{Presentations at Professional Meetings}

\begingroup
\let\clearpage\relax
\printbibliography[heading=none,	keyword=cfes2015, keyword=presentation]
\endgroup

\section{Technical Reports Documenting Applied Research in Support of the
Guam Coconut Rhinoceros Beetle Project}

\begingroup
\let\clearpage\relax
\printbibliography[heading=none,	keyword=cfes2015, keyword=crbtechreport]
\endgroup

\section{Guam New Invasive Species Alerts}

\begingroup
\let\clearpage\relax
\printbibliography[heading=none,	keyword=cfes2015, keyword=gisa]
\endgroup

\section{\label{sec:Web-Sites-Designed}Web Sites Designed and Maintained
by Me}

For the past five years, I have been searching for the ``right''
technology for providing on-line extension information. The features
I want include:
\begin{itemize}
\item Ease of use, including immediate, on-line editing, so that colleagues
and collaborators can create content
\item Ability to display digital images at several resolutions
\item Full text search
\item Methods for handling on-line and offline references
\item Fine grained security which protects client confidentiality and allows
for both protected, internal and public information sharing
\end{itemize}
My current technology of choice is Drupal, a free, open source contents
management system. 

\subsection{ANR Web Site.}

Home page: \url{http://guaminsects.net/anr}\\
This Drupal site is intended to facilitate sharing both internal and
external information generated by the Agriculture and Natural Resources
Unit of the University of Guam Cooperative Extension Service. This
site is currently being used heavily by the Guam CRB Eradication Project.
I also use this site for documenting ny diagnostics work. I provide
a recent example web page documenting discovery of thrips in anthurium
flowers. \\
(Evidence 2.5.1; available on-line at \url{http://guaminsects.net/anr/content/thrips-damaging-anthurium-flowers})

\subsection{Insects of Guam Web Site}

Home page: \url{http://guaminsects.myspecies.info}\textbf{}\\
This Drupal site is being evaluated for sharing information on Micronesian
insects. Information will include include specimen level information
from the UOG insect collection complete with digital images and literature
references. It was built using a template developed by the Scratchpad
project \url{http://scratchpads.eu/} is sponsored by the European
Institute of Distributed Taxonomy (EDIT) and the Natural History Museum
in London . The ScratchPad project is celebrating the International
Year of Biodiversity by highlighting a different Scratchpad taxon
every week. I was honored to have one of my pages, describing the
indigenous bug, \textit{Leptocoris vicinus}, highlighted during the
week of April 18 to 24, 2010.\\
(Evidence 2.5.2)

\subsection{Micronesia Biosecurity Plan Review Web Site}

Home page: \url{MBP.GuamInsects.net}\\
This is a secure, private Drupal site developed to facilitate sharing
information among those reviewing the Micronesia Biosecurity Plan.

\subsection{Moodle Site for my AG 109 Insect World Course}

Home page: \url{http://campus.uogdistance.com/course/view.php?id=286}\\
This site was my first experence with Moodle, a content management
system designed for teachers. I originally built it to provide on-line
resources for my students, but later decided to open a few wikis to
promote collaboration on laboratory exercises. I also kept track of
grades using Moodle. Examples from this site include the course resource
page (Evidence 2.5.3a; available on-line at \url{http://campus.uogdistance.com/mod/resource/view.php?id=7349})
and a small PHP program I wrote to facilitate printing pinned insect
specimen labels (Evidence 2.5.3b; available on-line at \url{http://tinyurl.com/insect-labels}).

\subsection{Knowledgebase Wiki for the UOG Cooperative Extension }

Home page: \url{http://www.guaminsects.net/uogces/kbwiki/index.php}
\textbf{}\\
This was my first attempt at building an extension website to facilitate
collaborative content creation. Digital copies of all of ANR's pest
fact sheets can be found on this site. There is also a list of insect
pests found on all major crops grown in Micronesia. I stopped maintaining
this site in May, 2009 because the ANR site built with Drupal has
more of the features I need.\\
(Evidence 2.5.4)

\subsection{Western Micronesia Regional Invasive Species Council Wiki}

Home page: \url{http://www.guaminsects.net/gisac/index.php}\\
Originally built for the Guam Invasive Species Advisory Council, this
site was quickly adopted fo sharing regional information on invasive
species by the Western Micronesia Regional Invasive Species Council.\\
(Evidence 2.5.5)

\subsection{Guam Insects Blog Site}

Home page:\url{ http://blog.guaminsects.net/} \\
I ran into recurring technical problems with this site which uses
the WordPress content management system and have mor or less abandonded
development and maintenance.

\subsection{Life Desk Site for Micronesian Insects}

Home page: \url{http://micronesianinsects.lifedesks.org/}\\
This site uses a Drupal template being developed by the Encyclopedia
of Life Project. I evaluate it for sharing information on Micronesian
insects, but decided that the Scratchpad template (number 2, above)
had a better feature set for what I wanted to do.


\section{Grants}

My active and pending grants are listed in table \ref{tab:crb_grants})
and table \ref{tab:Grants-other-than-CRB}. Fourteen support staff
positions have been supported, partially of fully, from my grants
during 2012 and 2013 (Table \ref{tab:Staff-support-by-my-grants}).
Two of my recent grant proposals were not funded (Table \ref{tab:Unfunded-grant-proposals.}).

\subsection{US Forest Survey Program Review}

During May 2014, officials from the US Forest Survey paid a visit
to Guam to review performance on grants they have given to UoG over
the past five years. A summary of my activities in support of these
grants is available online at \url{http://guaminsects.net/anr/content/materials-forest-service-review-team}

\pagebreak{}

\begin{table}[p]
\protect\caption{\label{tab:crb_grants}Active and pending grants in support of the
Guam Coconut Rhinoceros Beetle Eradication Project.}

\centering{}%
\begin{tabular}{>{\raggedright}p{5cm}>{\centering}p{3cm}>{\centering}p{3cm}>{\raggedleft}p{2cm}}
\hline 
Title &
Source &
Grant No.or UOG Account &
Amount\tabularnewline
\hline 
\hline 
Support for the Guam Coconut Rhinoceros Project  &
USDA Forest Service &
{\small{}11-DG-11052012-101} &
\$227,000\tabularnewline
Biological Control of the Coconut Rhinoceros Beetle &
USDA APHIS &
12-8515-1555-CA &
\$40,000\tabularnewline
Support for the Guam Coconut Rhinoceros Project  &
USDA Forest Service &
{\small{}11-DG-11052012-101} &
\$150,000\tabularnewline
Biological Control of the Coconut Rhinoceros Beetle {[}PENDING\} &
USDA APHIS &
 &
\$40,000\tabularnewline
\hline 
\end{tabular}
\end{table}

\pagebreak{}

\begin{longtable}{>{\raggedright}p{2in}>{\centering}p{1.25in}>{\centering}p{1.25in}>{\raggedleft}p{0.75in}}
\caption{\label{tab:Grants-other-than-CRB}Active and pending grants other
than those supporting the Guam Coconut Rhinoceros Beetle Eradication
Project.}
\tabularnewline
\endfirsthead
\midrule 
Title &
Source &
Grant No. or

UOG Account &
Amount\tabularnewline
\midrule 
National Plant Diagnostic Network (NPDN) &
via UC Davis &
201223902-09 &
\$7,550\tabularnewline
\midrule 
National Plant Diagnostic Network (NPDN) {[}PENDING{]} &
via UC Davis &
 &
\$7,550\tabularnewline
\midrule 
Octocula conservation {[}PENDING{]} &
USFWS via DAWR &
 &
\$20,000\tabularnewline
\bottomrule
\end{longtable}

\pagebreak{}

\begin{longtable}{>{\raggedright}p{2in}>{\centering}p{1.25in}>{\centering}p{1.25in}>{\raggedleft}p{0.75in}}
\caption{\label{tab:Unfunded-grant-proposals.}Recent unfunded grant proposals.}
\tabularnewline
\midrule 
Title &
Source &
Notes &
Amount\tabularnewline
\endfirsthead
\midrule 
Title &
Source &
Notes &
Amount\tabularnewline
\endhead
\midrule 
Guam Insect Biodiversity &
USFWS via DAWR &
 &
\$20,000\tabularnewline
\midrule 
Octocula conservation &
USFWS &
 &
\$18,000\tabularnewline
\bottomrule
\end{longtable}

\pagebreak{}

\begin{longtable}{r>{\raggedright}p{5cm}}
\caption{\label{tab:Staff-support-by-my-grants}Staff support by my grants.}
\tabularnewline
\endfirsthead
\midrule 
1 &
 Bob Bourgeois\tabularnewline
2 &
 Roger Brown (partially)\tabularnewline
3 &
 Roland Quitugua (partially)\tabularnewline
4 &
 Ian Iriarte\tabularnewline
5 &
 Vincent Benavente\tabularnewline
6 &
 John Diego\tabularnewline
7 &
 Ken Leon Guerrero\tabularnewline
8 &
 Roland Cabrera\tabularnewline
9 &
 Derrick Diego\tabularnewline
10 &
 Marty Hara\tabularnewline
11 &
 Ken San Nicolas\tabularnewline
12 &
 Jessica Gross\tabularnewline
13 &
Cris Crisostimo\tabularnewline
14 &
Raymondo San Miquel\tabularnewline
\bottomrule
\end{longtable}

\chapter{University and Community Service}

\section{Participation in the Good to Great Initiative}

\section{Teaching }

\subsection{AG-109 Insect World}

I taught this course four times. My score on the student evaluations
are consistently above average (Table \ref{tab:Student-evaluation}).

\begin{table}[h]
\protect\caption{\label{tab:Student-evaluation}Student evaluation for AG109, \emph{Insect World}.}
\centering{}%
\begin{tabular}{cccc}
\hline 
Term & My Evaluation & College Average & University Average\tabularnewline
\hline 
Fall 2009   & 3.659 & 3.565 & 3.552\tabularnewline
Spring 2011 & 3.986 & 3.519 & 3.617\tabularnewline
Spring 2012 & 3.863 & 3.570 & 3.612\tabularnewline
Spring 2013 & na    & na    & na   \tabularnewline
Fall 2014 	& na    & na    & na   \tabularnewline   
\hline 
\end{tabular}
\end{table}

\subsection{AG/BIO-345 General Entomology}

\begin{table}[h]
\protect\caption{\label{tab:Student-evaluation}Student evaluation for AG/BIO-345, \emph{General Entomology}.}
\centering{}%
\begin{tabular}{cccc}
\hline 
Term & My Evaluation & College Average & University Average\tabularnewline
\hline 
Fall 2013 & 3.875 & 3.522 & 3.586\tabularnewline
\hline 
\end{tabular}
\end{table}

\section{Music}

As an amateur horn player I play regularly, and often very badly,
with the Guam Symphony Orchestra and occasionally with the Guam Territorial
Band. I have played for UOG graduations and for concerts arranged
by the UOG music department. In spring 2013, I played in a horn quartet
in a UOG Music Department recital.

\section{Conference on Island Sustainability}

I served on the planning committee for the Island Conference on Island
Sustainability in 2010, 2011 and 2012.

\section{Micronesia Biosecurity Plan}

I have been involved with the Micronesia Biosecurity Plan (MBP) since
its inception. The MBP is being developed to mitigate an expected
increase in invasive species associated with the Guam military buildup.
A first draft of the MBP was written by federal agencies supported
by a \$2.7M grant from the Department of Defense (DoD). In January,
2010, I made 2 presentations at an an MBP organizational meeting:
\emph{Biological Invasion of Guam} and \emph{Invasive Insects on Guam.}
In 2011, I assisted the UOG Center for Island Sustainability in securing
a DoD cooperative agreement (CA) which provides \$1.1M to UOG to provide
a peer review of the MBP and to develop an implementation plan. I
was named as CoPI with Dr. Frank Camacho on the original CA, but I
have resigned from this position to concentrate more on my entomological
interests. However, I am still involved as a reviewer and I recently
helped out by building a private, secure website to facilitate sharing
information among thoe working on the MBP.

\section{Collaboration on CESU Rare Butterfly and Snails Survey Grant}

I am collaborating with Dan Lindstrom, John Benedict, Frank Camacho,
and Curt Fiedler (UOG Biology), Alex Kerr (UOG Marine Lab), Brent
Holland and Dan Rubinoff (UH Manoa) on a DOD funded survey of rare
butterflies and snails. My contribution is a literature review of
\emph{Hypolimnas octocula marianensis} for publication in Micronesica,
design and maintenance of project website and development of butterfly
camera traps.

\section{Collaboration on Biocontrol of Cycad Aulacaspis Scale}

I am working with Tom Marler on introduction of parasitoids for biocontrol
of the \emph{Aulacaspis yasumatsui}.

\section{Collaboration on EPSCOR Proposal}

I have submitted many ideas to be incorported into the EPSCOR proposal.
I spent one and a half days at the PACSTEM meeting at the Hyatt discussing
some of these ideas with colleagues.

\section{University Technical Advisory Committee}

I serve on UTAC as the representative for the College of Natural and
Applied Sciences.

\section{Undergraduate Curriculum Review Committee (UCRC)}

In the April 2013 Faculty Elections, I was elected to serve on the
UCRC.

\end{document}
